%!TEX root = Thesis.tex

\chapter{Conclusion}
\chaptermark{Conclusion}
\label{ch:conclusion}

%- General intro - why is this interesting

The \tBuxantphos{} ligand has been tested in a number of catalytic reactions with varying results.\cite{Mispelaere2005, Dongol2007, Ohshima2009, Cabello2007, Zhan2012, Ashcroft2013, Behr2013, Friis2014, Raoufmoghaddam2013}  The conversions are typically very different to those obtained with \Phxantphos{} suggesting differences in their coordination behaviour.  However, few studies had investigated the coordination chemistry of the \tBuxantphos{} ligand.  This thesis aimed to address that deficit by synthesising two new \tBuxantphos{} ligands, and investigating the properties and coordination chemistry of the three ligands, with a particular focus on transition metals used in catalysis.  

%The xantphos class of ligands have been studied to investigate the impact of subtle changes in their bite-angle on the coordination chemistry and catalytic activity of their complexes.  Changing the bridging group in the \Phxantphos{} ligands has been shown to have an effect on the catalytic activity of their late-transition metal complexes.  Recently the \tBuxantphos{} ligand has gained attention in the literature.  \tBuXantphos{} has been tested as an ancillary ligand in a range of different catalytic reactions with varying results.  The outcomes were typically drastically different to those for \Phxantphos{} indicating a potential difference in the coordination chemistry of the two ligands.  Prior to the start of this research the only complexes of \tBuxantphos{} were [Au(\tBuxantphos)][Au\ce{X2}] (X = Cl, Br, I).  Since then other groups have published a palladium complex featuring a monodentate \tBuxantphos{} ligand and a number of rhodium complexes.  An X-ray crystal structure of \trans{}-[Pd(\tBuxantphos)\ce{Cl2}] had also been reported to the \gls{CSD}.  However, while interesting, these studies have not given much insight into the changes that occur as a result of the large bite-angle of \tBuxantphos{}.  Furthermore, changing the bridging group in \Phxantphos{} was shown to have a significant impact on the catalytic activity of its complexes.  The current study has investigated three \tBuxantphos{} ligands, with different bridging groups, in order to gain an understanding of the coordination chemistry of the \tBuxantphos{} ligands including the changes that can result from small differences in the bite-angles despite all of the ligands having very large bite-angles.

Two new \tBuxantphos{} ligands with a \ce{SiMe2} (\tBusixantphos) or a sulfur atom (\tButhixantphos) in place of the \ce{CMe2} bridging group in \tBuxantphos{} were synthesised.  The synthetic method for \tBusixantphos{} and \tButhixantphos{} was also used successfully to synthesise \tBuxantphos{}.  This synthesis is advantageous as the only by-products are the monophosphines, which can be re-used in a further cycle to produce additional diphosphine.  The bite-angles of the \tBuxantphos{} and \Phxantphos{} were calculated using \gls{DFT} methods with the \tBuxantphos{} ligands found to have larger bite-angles (126.80--127.56\degrees{}) than the \Phxantphos{} ligands (111.89--114.18\degrees{}).  The values calculated for the \Phxantphos{} ligands using DFT are consistent with the literature values calculated using molecular mechanics.  

The \tBuxantphos{} ligands are both Br\o nsted and Lewis bases.  The three ligands reacted rapidly with strong acids forming (\tBuxantphos)\ce{H+} with a single proton that exchanged rapidly between the two phosphorus atoms at room temperature.  This process has a coalescense temperature of around -40 \degC{} in solution.  Synthesis of the \tBuxantphos{} selenides and analysis of their \JPSe{} values gave \pKb{} values of 5.67 for \tBusixantphos, 6.90 for \tButhixantphos{}, and 6.72 for \tBuxantphos{}.  These are much lower than the value for \Phxantphos{} of 13.55, as expected due to greater electron density on the phosphorus atoms in \tBuxantphos{} resulting from the \tBu{} substituents.  

The coordination chemistry of the \tBuxantphos{} ligands was investigated with silver, rhodium, platinum, and palladium.  Two types of silver complexes were reported, [Ag(\tBuxantphos)Cl] and [Ag(\tBuxantphos)]\ce{BF4}.  Both are monomeric despite the free-coordination site in [Ag(\tBuxantphos)]\ce{BF4}.  The \tButhixantphos{} ligand in the x-ray crystal structure of [Ag(\tButhixantphos)Cl] had a bite-angle of 130.50(7)\degrees{}, which is larger than the bite-angle in the previously reported [Ag(\Phxantphos)Br] complex (109.37(1)\degrees{}).  Both of these are close to the natural bite-angle of the xantphos ligand, indicating that the coordination geometry around the silver is controlled by the diphosphine ligand.  Although the crystal structure of [Ag(\tButhixantphos)Cl] suggested that the \tBu{} substituents should have different NMR signals, this was not observed spectroscopically, likely due to the rapid inversion of the xantphos backbone.  

Rhodium complexes are used as catalysts for hydrogenation and hydroformylation which are performed under hydrogen or a mixed hydrogen/carbon monoxide atmosphere.  [Rh(\tBuxantphosk)Cl] complexes were formed for all three \tBuxantphos{} ligands. The reactivity of these complexes towards \ce{H2}, CO, and \ce{O2} was investigated, forming [Rh(\tBuxantphosk)Cl(\ce{H)2}], [Rh(\tBuxantphos)(\ce{CO)2Cl]}, and [Rh(\tBuxantphosk)Cl(\hapto{2}-\ce{O2})] complexes. The value of the \JRhH{} coupling constant for the hydride \trans{} to the oxygen atom in the [Rh(\tBuxantphosk)Cl(\ce{H)2}] complexes was largest for the \tBusixantphos{} complex, followed by \tButhixantphos{} then \tBuxantphos{}.  This shows that the ligands have differing \trans{} influences; \tBusixantphos{} \textless{} \tButhixantphos{} \textless{} \tBuxantphos.  The X-ray crystal structure of [Rh(\tBuxantphosk) Cl(\hapto{2}-\ce{O2})] was disordered with the dioxygen ligand replaced in 15\% of the sites by an oxo.  This is the first crystallographic evidence for a rhodium(III) oxo complex, and only the third rhodium oxo complex that has been reported.  Attempts to synthesis the oxo complex directly were promising, with a new peak present in the \phosphorus{} NMR spectrum for a short-lived species.  However, further research and characterisation is necessary.

Upon reaction of \Phthixantphos{} with [Pt(nb\ce{)3}] or [Pt(\ce{C2H4)3}] the major product was [Pt(\Phthixantphos\ce{)2}]; this was the only product observed in a 2:1 reaction, and was characterised crystallographically.  The coordination chemistry with platinum(0) and palladium(0) showed some differences between the three \tBuxantphos{} ligands.  \tBuThixantphos{} reacts with [Pt(nb\ce{)3}] to form a mixture of [Pt(\tButhixantphos)] and [Pt(\tButhixantphos)(nb)].  Analogues of these complexes were formed in the reaction between [Pt(nb\ce{)3}] and \tBusixantphos{}.  However, over time a complex with NMR data consistent with platinum(II) is observed.  Using \tBuxantphos{} small amounts of [Pt(\tBuxantphos)] forms, but the final product is [Pt(\tBuxantphos)H]X, possibly through reaction of the 14-electron complex with a component of the reaction mixture.  The ability to isolate [Pt(\tButhixantphos)], indicates the protection that the wide bite-angle and size of the \tBu{} substituents can impart to a metal centre.  Isolable complexes of the type [Pt(diphosphine)] are rare, although examples with monophosphines are known.  Reaction of the \tBuxantphos{} ligands with [Pd(nb\ce{)3}] formed [Pd(\tBuxantphos)] and [Pd(\tBuxantphos)(nb)] complexes for all three ligands.  [Pt(\tBuxantphos)(\ce{C2H4})] complexes \emph{via} reaction with [Pt(\ce{C2H4)3}].  [Pd(\tButhixantphos)], [Pt(\tButhixantphos)], and [Pt(\tButhixantphos)(\ce{C2H4})] all react with oxygen.  On palladium the oxygen is readily removed \emph{in vacuo}, whereas the platinum complex does not lose oxygen under vacuum.  [Pt(\tButhixantphos)(\hapto{2}-\ce{O2})] is unreactive towards \ce{C2H2}, \ce{H2}, \ce{CO2}, \ce{NH4PF6} and pta.  However, CO can insert into the O-O bond forming a carbonate, complex which progresses through two intermediates resulting in [Pt(\tButhixantphos-H-\dento{}-\emph{C,P,P}\textprime)OH].  This reactivity is unprecedented in the literature.  

The coordination chemistry of the \tBuxantphos{} ligands with platinum(II) and palladium(II) starting materials was also explored.  Regardless of the geometry of the starting material exclusively \trans{}-[M(\tBuxantphos)\ce{Cl2}] (M = Pd, Pt) complexes were formed.  The reaction of \tBusixantphos{} with [Pt\ce{Cl2}(hex)] was hindered due to the competing protonation of the ligand due to the higher basicity of \tBusixantphos{} compared to \tButhixantphos{} and \tBuxantphos{}.  The X-ray crystal structure of [Pt(\tBuxantphos)\ce{Cl2}] is unique among [Pt\ce{Cl2}(PP)] complexes, with a P-Pt-P angle of 151.722(15)\degrees{}, with no similar angles reported in the \gls{CSD}.  The platinum dichloride complexes underwent solvent-dependent dissociation of a chloride ligand, forming [Pt(\tBuxantphosk)Cl\ce{]+} in \ce{CDCl3} and \ce{CD2Cl2}.  The solvent-dependent behaviour was not observed for the palladium complexes.  Computational investigations showed that, although the pincer complexes are consistently higher in energy than the dichloride complexes, the energy difference is much lower in polar solvents.  Entropy would also promote  the dissociation of the chloride ligand leading to the experimentally observed spontaneity.  No reaction occurred between \tButhixantphos{} and [Pt(hex)\ce{Me2}], likely due to the stronger coordination of the methyl ligands.  The reactions of the \tBuxantphos{} ligands with [PtCl(hex)Me] formed only [Pt(\tBuxantphos)Me\ce{]+} pincer complexes with no chlorido-methyl species observed.  This is likely the result of the high \trans{}-influence of the methyl ligand promoting loss of the chloride.  Computational investigation showed that the fourth coordination site is occupied by the oxygen in the backbone of the \tBuxantphos{} ligands.  The value of the \JPtC{} coupling constant is largest for \tBusixantphos{}, then \tButhixantphos{}, then \tBuxantphos{} consistent with the O-\trans{} influence series observed in the rhodium complexes; \tBusixantphos{} \textless{} \tButhixantphos{} \textless{} \tBuxantphos.  

Overall this thesis provides an account into the synthesis, properties and coordination chemistry of \tBusixantphos, \tButhixantphos{}, and \tBuxantphos{} with four late-transition metals; silver, rhodium, platinum, and palladium.  The ligands showed: formation of monomeric silver complexes with \dento{}-\emph{P,P}\textprime{} coordination; rhodium complexes with meridional \POP{} coordination in square-planar, octahedral and trigonal bipyramidal complexes, and \dento{}-\emph{P,P}\textprime{} coordination in trigonal bipyramidal species; platinum(0) and palladium(0) complexes with \dento{}-\emph{P,P}\textprime{} coordination in the presence and absence of other ligands and; exclusive \trans{}-\dento{}-\emph{P,P}\textprime{}coordination to platinum(II) and palladium(II) and \POP{} bonding modes.  The final mode observed was the metallation of the \tBu{} group on platinum to form a \dento{}-\emph{C,P,P}\textprime{} ligand.  X-ray crystallography showed that the \tButhixantphos{} ligand can achieve \dento{}-\emph{P,P}\textprime{} coordination with bite-angles ranging from 117--151\degrees{}.  

%Ligands chapter
%- Synthesis of the new ligands - including a new method of forming \tBuxantphos
%- Bite-angle calculations - use of DFT
%- Basicity
%	- Selenides pKb stuff
%Silver
%- Two complexes
%- Both monomeric
%- Crystal structure bite-angle close to natural
%Rhodium
%- Pincer chloride complexes
%- Split hydrogen
%- Different \trans{} influence of the ligands
%- Form carbonyl complexes
%- Dioxygen and oxo
%	- First rhodium(III) oxo
%- Pincer in square-planar and trigonal bipyramidal structures
%Platinum and palladium(0)
%- Different reactivity with the nb complexes
%- Formation of ethene species
%- Synthesis of dioxygen complexes
%	- Reversible on Pd not on Pt
%- Reaction of pt(tButhixantphos)O2 with CO
%- Computational results
%Platinum(II)
%- Formation of exclusively trans Pt and Pd dichlorides
%- Solvent dependent dissociation of a chloride ligand
%	- Computational results
%- Formation of the methyl pincer immediately
%	- Computational results
%- Difference in the \trans{} influence of the ligands
