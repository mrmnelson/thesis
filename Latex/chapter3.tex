%!TEX root = Thesis.tex

\chapter{The vuwthesis.cls File}
\label{ch:thesiscls}

OK, this is big file and I don't even pretend to understand all of it.  It is based upon the report class which is supplied with \LaTeX{} but it is heavily edited and expanded upon to do everything I needed for the thesis.  I will only focus upon the bits that are of any particular relevance.

\small\singlespacing
\begin{verbatim}
\NeedsTeXFormat{LaTeX2e}
\ProvidesClass{thesis}

\DeclareOption*{\PassOptionsToClass{\CurrentOption}{report}}
\ProcessOptions

\LoadClass{report}
\end{verbatim}

\normalsize\doublespacing
Some introductory stuff which we'll ignore.

\small\singlespacing
\begin{verbatim}
% conditionals for this class file
\RequirePackage{ifpdf}
% default language
\RequirePackage[english]{babel}
% glossary package
\RequirePackage[number=none,toc=true]{glossary}
% prevent T1 encoding error
\RequirePackage[T1]{fontenc}
% line spacing
\RequirePackage{setspace}
% page geometry
\RequirePackage{geometry}
\ifpdf
	% figures package [for pdflatex]
	\RequirePackage[pdftex]{graphicx}
\else
	% figures package [for latex]
	\RequirePackage[dvips]{graphicx}
\fi
% chemical formulas
\RequirePackage[version=3]{mhchem}
% numbered structures
\RequirePackage{chemcompounds}
% color support
\RequirePackage{color}
% row spanning in tables
\RequirePackage{multirow}
% ACS referencing
\RequirePackage[numbers,sort&compress,super]{natbib}
% times font
\RequirePackage{times}
% tables with footnotes
\RequirePackage{threeparttable}
% sideways tables
\RequirePackage{rotating}
% new floats and caption layout
\RequirePackage{caption}
% subfigures
\RequirePackage{subfigure}
% slanted greater/less than or equals to
\RequirePackage{amssymb}
% source code support
\RequirePackage{listings}
% modification of labels in schemes
\RequirePackage[floats=float]{chemscheme}
\ifpdf
	% eps to pdf conversion on the fly [for pdflatex]
	\RequirePackage{epstopdf}
	% hyperlinks [for pdflatex]
	\RequirePackage[pdftex]{hyperref}
\else
	% hyperlinks [for latex]
	\RequirePackage[dvips]{hyperref}
\fi
\end{verbatim}

\normalsize\doublespacing
This tells the compiler which packages are needed to compile this documents of this class.  Each of these has been added for a reason and with the possible exception of the listings package, you will likely need them all.  Depending on your version of \LaTeX{} this may or may not pose problems.  MiKTeX for Windows users shouldn't have any problems because the update mechanism is so good.  Linux users may have to manually upgrade to the latest versions of some of these packages on their own. For the epstopdf package to work, Windows users will need to add ``\verb$--$enable-write18'' to the command line options for pdflatex in the ``Define output profiles'' option of TeXnicCenter.

\small\singlespacing
\begin{verbatim}
% Add a \textsubscript command equivalent to \textsuperscript
% From subscript.sty code fragment on CTAN
\DeclareRobustCommand*\textsubscript[1]{%
  \@textsubscript{\selectfont#1}}
\def\@textsubscript#1{%
  {\m@th\ensuremath{_{\mbox{\fontsize\sf@size\z@#1}}}}}
\end{verbatim}

\normalsize\doublespacing
Believe it or not, \LaTeX{} does not include a textsubscript command to complement its textsuperscript command.  This adds one.

\small\singlespacing
\begin{verbatim}
\lstset{basicstyle=\footnotesize\ttfamily}
\lstset{breaklines=true,showstringspaces=false}
\lstset{tabsize=3,breakindent=15pt}
\lstset{numbers=left,numberstyle=\tiny,stepnumber=5,
   numbersep=5pt,firstnumber=1}
\lstloadlanguages{[Visual]C++,Matlab}
\end{verbatim}

\normalsize\doublespacing
Some basic settings for the listings package.

\small\singlespacing
\begin{verbatim}
% use a wider glossary than the default 0.6\linewidth
\descriptionwidth=0.8\linewidth
% stop the glossary from grouping alphabetically
\setglossary{delimT={\cr & \cr},gloskip={}}
\end{verbatim}

\normalsize\doublespacing
Set up the glossary package.

\small\singlespacing
\begin{verbatim}
\geometry{hmargin={4cm,2cm},vmargin={1.5cm,2cm}}
\setlength{\parindent}{0pt}
\setlength{\parskip}{14pt plus 2pt minus 1pt}
\end{verbatim}

\normalsize\doublespacing
Here is where we define our margins, indentation and paragraph skip distance.  To prevent indenting at all, the indent length is set to 0 points.

\small\singlespacing
\begin{verbatim}
\def\@biblabel#1{#1.}
\end{verbatim}

\normalsize\doublespacing
Change the default labelling of reference entries.

\small\singlespacing
\begin{verbatim}
\def\subject#1{\gdef\@subject{#1}}
\def\degree#1{\gdef\@degree{#1}}
\def\institution#1{\gdef\@institution{#1}}
\def\logo#1{\gdef\@logo{#1}}

\renewcommand\maketitle{\begin{titlepage}%
	\let\footnotesize\small
	\let\footnoterule\relax
	\let \footnote \thanks
	\begin{center}\begin{large}%
		{\ } \par
		\vspace{1.5cm}
		{\huge\bfseries\@title\par}%
		\vspace{2cm}
		{by\\ \@author}\par
		\vspace{0.08\textheight}
		\includegraphics[height=4cm]{\@logo}\par
		\vspace{0.06\textheight}
		{A thesis\\ submitted to \@institution\\
		 in fulfilment of the\\requirements for the degree of\\
		 \@degree\\in \@subject.\\ \vspace{2cm}
		 \@institution\\ \@date \par}%
	\end{large}\end{center}\par
\end{titlepage}
\setcounter{footnote}{0}%
\global\let\thanks\relax
\global\let\maketitle\relax
\global\let\@thanks\@empty
\global\let\@author\@empty
\global\let\@date\@empty
\global\let\@title\@empty
\global\let\title\relax
\global\let\author\relax
\global\let\date\relax
\global\let\and\relax
}
\end{verbatim}

\normalsize\doublespacing
This complicated beast makes our title page.  Depending on the length of the title of your thesis, you may need to jigger around with the vertical spacings between the title and the logo etc.

\small\singlespacing
\begin{verbatim}
\renewcommand{\@makechapterhead}[1]{
   {
      \normalfont \centering \bfseries \Large
      \ifnum \c@secnumdepth >\m@ne
         \begin{itshape}
            \@chapapp\space\thechapter\par
         \end{itshape}
      \fi
      #1\par
      \vspace{0.5cm}%
   }
}

\renewcommand{\@makeschapterhead}[1]{
   {
      \normalfont \centering \bfseries \Large
      #1\par
      \vspace{0.5cm}%
   }
}
\end{verbatim}

\normalsize\doublespacing
This redefines the chapter headings.

\small\singlespacing
\begin{verbatim}
\renewcommand{\contentsname}{Table of Contents}
\renewcommand{\bibname}{References}
\end{verbatim}

\normalsize\doublespacing
And this names the contents pages and the bibliography appropriately.

\small\singlespacing
\begin{verbatim}
% Redefine the default placement of figures and table as bp (not
% tbp). I like figures/tables etc only to occur AFTER they are
% mentioned in the text. Because the chemscheme package doesn't
% initialise the scheme environment's placement until after
% \begin{document}, the redefinition of the scheme's placement
% must occur in thesis.tex.
\floatplacement{figure}{bp}
\floatplacement{table}{bp}
\end{verbatim}

\normalsize\doublespacing
Hopefully the comment is self-explanatory.  Floats can be placed htbp (here, top, bottom or page).  The \LaTeX{} default of tp often leads to figures going at the top of the page on which they are first referenced (ie. before the text) which I disliked.

\small\singlespacing
\begin{verbatim}
\newfloat{structure}{hbp}{lox}[chapter]
\end{verbatim}

\normalsize\doublespacing
Sets up structure as a new float type.  Schemes are already added as a float by the chemscheme package.

\small\singlespacing
\begin{verbatim}
\captionsetup{labelfont=bf,labelsep=space,
   justification=centering}
\end{verbatim}

\normalsize\doublespacing
Redefines the caption layout so ``Figure x.y'' is in bold and the rest of the caption is not.

\small\singlespacing
\begin{verbatim}
% Shortcuts for dH, dC and dN
\newcommand{\dH}{$\delta$\textsubscript{H}~}
\newcommand{\dC}{$\delta$\textsubscript{C}~}
\newcommand{\dN}{$\delta$\textsubscript{N}~}

% Shortcuts for 1H, 13C and 15N
\newcommand{\proton}{\ce{^{1}H}}
\newcommand{\carbon}{\ce{^{13}C}}
\newcommand{\nitrogen}{\ce{^{15}N}}

% Shortcuts for H2 and H3
\newcommand{\Htwo}{H\textsubscript{2}}
\newcommand{\Hthree}{H\textsubscript{3}}

% Shortcut for 1JCH etc.
\newcommand{\oneJCH}{\mbox{\textsuperscript{1}
   \emph{J}\textsubscript{CH}}}
\newcommand{\oneJNH}{\mbox{\textsuperscript{1}
   \emph{J}\textsubscript{NH}}}
\newcommand{\oneJXH}{\mbox{\textsuperscript{1}
   \emph{J}\textsubscript{XH}}}

% Shortcuts for [M + H]+, [M + Na]+, [M + NH4]+,
% [M - H]-, [M - Na]- and [M - 2Na]2-
\newcommand{\MplusH}{\mbox{[M + H]\textsuperscript{+}}}
\newcommand{\MplusNa}{\mbox{[M + Na]\textsuperscript{+}}}
\newcommand{\MplusAmmonium}{\mbox{[M + \ce{NH4}]
   \textsuperscript{+}}}
\newcommand{\MminusH}{\mbox{[M $-$ H]$^{-}$}}
\newcommand{\MminusNa}{\mbox{[M $-$ Na]$^{-}$}}
\newcommand{\MminusNaNa}{\mbox{[M $-$ 2Na]
   \textsuperscript{2}$^{-}$}}

% Shortcuts for lambda max and nu max
\newcommand{\lambdamax}{\mbox{$\lambda$\textsubscript{max}}}
\newcommand{\numax}{\mbox{$\nu$\textsubscript{max}}}

% Shortcut for reciprocal centimetres
\newcommand{\percm}{\mbox{cm$^{-}$\textsuperscript{1}}}

% Shortcut for degrees C
\newcommand{\degC}{\mbox{$\,^\circ$C}}

% Shortcuts for ED50, IC50, LC50 and LD50
\newcommand{\EDfifty}{\mbox{ED\textsubscript{50}}}
\newcommand{\ICfifty}{\mbox{IC\textsubscript{50}}}
\newcommand{\LCfifty}{\mbox{LC\textsubscript{50}}}
\newcommand{\LDfifty}{\mbox{LD\textsubscript{50}}}
\end{verbatim}

\normalsize\doublespacing
Here are a whole pile of newly defined commands to save typing later and to try and ensure consistency throughout the document.  Note that the ``\small\verb$~$\normalsize'' symbol forces a space after the command.  If you require a space after the other commands, you should follow them with double braces.  For example, the command ``\small\verb$\dH$\normalsize'' produces ``\dH'' (note the space at the end).  The command ``\small\verb$\proton spectra$\normalsize'' produces ``\proton spectra'' while ``\small\verb$\proton{} spectra$\normalsize'' produces ``\proton{} spectra'' which is probably what you want.  The ``\small\verb$\degC$\normalsize'' command produces a half-space followed by the degrees Centigrade symbol so you can just use ``\small\verb$20\degC$\normalsize'' to produce ``20\degC''.

\small\singlespacing
\begin{verbatim}
% Try and limit hyphenation
\hyphenpenalty=5000
\tolerance=1000
\end{verbatim}

\normalsize\doublespacing
Some commands to try and avoid too much hyphenation.

\small\singlespacing
\begin{verbatim}
% Standard typesetting for fields
\newcommand{\field}[1]{\textbf{#1}}
% and values of database entries
\newcommand{\entry}[1]{\textit{#1}}
\end{verbatim}

\normalsize\doublespacing
New commands to simplify the field and entry terms in my database chapter.

\small\singlespacing
\begin{verbatim}
\newcommand{\approximately}{ca.}
\end{verbatim}

\normalsize\doublespacing
By defining a command for approximately, I could change all instances at once if I needed to.

\small\singlespacing
\begin{verbatim}
% Use a,b,c etc. for footnotes to avoid confusion with \cite
\renewcommand{\thefootnote}{$\fnsymbol{footnote}$}
\end{verbatim}

\normalsize\doublespacing
Change footnotes to letters by default.  Numbers are obviously ambiguous with citations.

\small\singlespacing
\begin{verbatim}
\newcommand{\fixme}[1]{\colorbox[rgb]{1,0.5,0}{\textbf{#1}}}

\endinput
\end{verbatim}
\normalsize\doublespacing
The fixme command can be used to highlight stuff you still need to fix up (eg. ``\small\verb$\fixme{Don't forget to reference}$\normalsize'' produces ``\fixme{Don't forget to reference}'').  You can change the colour of the box by tailoring the rgb numbers to your liking.  Note that text in the fixme environment won't wrap to fit lines so either keep your comments short or use multiple fixme environments.

And that concludes the vuwthesis.cls file.  Now we'll look at some specific examples of how to put it all into use.
